\documentclass[12pt, oneside]{article}


\usepackage{listings}
\usepackage{fancybox,geometry,array,graphicx,fancyhdr,color,xspace,amsmath,
amssymb,amsfonts,subfigure,algorithm,algorithmic,hyperref,listofsymbols,tabularx,float}
\usepackage{lastpage}
\usepackage[table,xcdraw]{xcolor}

\usepackage{color}

\usepackage{xepersian}

\definecolor{dkgreen}{rgb}{0,0.6,0}
\definecolor{gray}{rgb}{0.5,0.5,0.5}
\definecolor{mauve}{rgb}{0.58,0,0.82}

\lstset{frame=tb,
  language=C++,
  aboveskip=3mm,
  belowskip=3mm,
  showstringspaces=false,
  columns=flexible,
  basicstyle={\small\ttfamily},
  numbers=none,
  numberstyle=\tiny\color{gray},
  keywordstyle=\color{blue},
  commentstyle=\color{dkgreen},
  stringstyle=\color{mauve},
  breaklines=true,
  breakatwhitespace=true,
  tabsize=3
}


\hypersetup{colorlinks=true,linkcolor=blue,citecolor=blue,filecolor=magneta,urlcolor=cyan}

\geometry{a4paper, tmargin=3cm, bmargin=2.5cm, lmargin=2.5cm, rmargin=2.5cm, footskip=1.5cm}
\renewcommand{\baselinestretch}{1.9} 


\settextfont{B Nazanin} 
\setlatintextfont[Scale=0.9]{Times New Roman} 
\setdigitfont{B Nazanin}

\newsavebox\mybox
\newenvironment{myfbox}{%
\begin{lrbox}{\mybox}%
\begin{minipage}{\dimexpr(\textwidth-2\fboxsep-2\fboxrule)}
\begin{flushleft}%
\begin{latin}%
\begin{quote}%
}{%
\end{quote}%
\end{latin}%
\end{flushleft}%
\end{minipage}%
\end{lrbox}%
\vskip10pt
\noindent
\fbox{\usebox\mybox}%
\vskip10pt
}
\newenvironment{myshadowbox}{%
\begin{lrbox}{\mybox}%
\begin{minipage}{\dimexpr(\textwidth-2\fboxsep-2\fboxrule-\shadowsize)}
}{%
\end{minipage}
\end{lrbox}%
\vskip10pt
\noindent
\shadowbox{\usebox\mybox}%
\vskip10pt
}









\begin{document}

\pagestyle{empty}

\vspace*{5cm}

\begin{center}
\noindent\fbox{%
    \parbox{\textwidth}{%
		\begin{center}			
				\vspace{5mm}
			       \textbf{مستندات}
				\\
				\vspace{5mm}

				\textbf{\Large{طراحی پایپلاین \lr{CI/CD}}}
				\\
				\vspace{2mm}
		\end{center}
    }%
}
\end{center}

\vspace*{5cm}

\begin{table}[h]
\centering
\begin{tabular}{|
>{\columncolor[HTML]{C0C0C0}}c |c|}
\hline
تاریخ تنظیم & ۱۳۹۹/۰۷/۲۰ \\ \hline
نسخه        & 1.0        \\ \hline
\end{tabular}
\end{table}



\cleardoublepage
\thispagestyle{empty}
\begin{center}
		به نام خدا	
\end{center}
\vspace{\fill}

\cleardoublepage

\tableofcontents
\cleardoublepage

\pagestyle{fancy}
\rhead{مستندات طراحی پایپلاین \lr{CI/CD}}


\newpage
\section{مقدمه}
فرآیند 
\lr{CI/CD}
یا
\lr{Continuous Integration}
و
\lr{Continuous Delivery}
با هدف خودکارسازی فرایند چرخه سورس‌کد به محصول پیاده‌سازی می‌شود.
\lr{Continuous Integration}
 می‌تواند مراحل مختلفی را
مانند 
\lr{Build}
،
\lr{Test}
،
بررسی کیفیت و ...
 شامل شود.
در مورد
 \lr{Continuous Delivery}
 می‌توان گفت که پس از مراحل
\lr{CI}
یک نسخه پایدار به تیم اجرا تحویل داده می‌شود.
از طرف دیگر
\lr{Continuous Deployment}
 زمانی اطلاق می‌شود که همه مراحل ذکرشده انجام شود و سپس در محیط عملیاتی
(\lr{Production})
«مستقر»
شود.

در این نوشتار سعی شده است
پایپلاین
\lr{CI/CD}
برای
یک سورس 
ساده
\lr{PHP}
 با استفاده از 
\lr{Docker}
و
\lr{Gitlab CI/CD}
انجام شود.

طراحی با این پیش‌فرض انجام‌ شده است که 
کانفیگ‌های
سرور
\lr{Build}
و
\lr{Gitlab Runner}
وجود دارد.

\section{سورس‌کد}
ابتدایی‌ترین نیاز برای طراحی پایپلاین، استفاده از کنترل ورژن است. در این مستند از ورژن کنترل
\lr{git}
و
زیرساخت 
\lr{Gitlab}
استفاده می‌شود.



\section{داکرایز}
برای 
\lr{Dockerize}
کردن پروژه،
\texttt{Dockerfile}
بشرح زیر تعریف می‌شود:

\begin{latin}
\begin{lstlisting}[language=Bash]
FROM php:7.4-apache 
COPY . /var/www/html
VOLUME [ "/var/www/html/config" ]
VOLUME [ "/var/www/html/uploads" ]
RUN chown -R www-data:www-data /var/www/html
\end{lstlisting}
\end{latin}


\section{کانفیگ \lr{Gitlab CI/CD}}
برای کانفیگ
\lr{Gitlab CI/CD}
 لازم است فایلی با 
 عنوان\footnote{امکان تغییر نام در تنظیمات پروژه وجود دارد.}
 \texttt{.gitlab-ci}
 در پروژه قرارگیرد.
همچنین می‌توان کانفیگ‌های مختلفی را به ازای هر برنچ استفاده کرد. 

فایل 
 \texttt{.gitlab-ci}
 در فرمت
 \lr{YAML}
 تعریف می‌شود.
 
 
\begin{latin}
\begin{lstlisting}
stages:
  - test
  - build
  - deploy
test:
  image: php:7.4
  stage: test
  before_script:
  - curl --location --output /usr/local/bin/phpunit https://phar.phpunit.de/phpunit.phar
  - chmod +x /usr/local/bin/phpunit
  script:
    - phpunit --configuration phpunit.xml
build:
  image: docker
  stage: build
  before_script:
    - docker login $CI_REGISTRY -u $CI_REGISTRY_USER -p $CI_REGISTRY_PASSWORD
  script:
    - docker build -t $CI_REGISTRY_IMAGE .
    - docker push $CI_REGISTRY_IMAGE
# Ref: https://docs.gitlab.com/ee/ci/ssh_keys/
deploy:
  image: alpine
  stage: deploy
  before_script:
    - apk add --no-cache openssh
    - echo "$PROD_SSH_PRIVATE_KEY" | tr -d '\r' | ssh-add -
    - mkdir -p ~/.ssh
    - chmod 700 ~/.ssh  
    - ssh-keyscan $PROD_SSH_ADDRESS >> ~/.ssh/known_hosts
    - chmod 644 ~/.ssh/known_hosts
    - docker login $CI_REGISTRY -u $CI_REGISTRY_USER -p $CI_REGISTRY_PASSWORD
  script:
    - ssh PROD_SSH_USER@$PROD_SSH_ADDRESS "docker login $CI_REGISTRY -u $CI_REGISTRY_USER -p $CI_REGISTRY_PASSWORD"
    - ssh PROD_SSH_USER@$PROD_SSH_ADDRESS "docker pull $CI_REGISTRY_IMAGE"
    - ssh PROD_SSH_USER@$PROD_SSH_ADDRESS "docker stop web-app && docker rm web-app"
    - ssh PROD_SSH_USER@$PROD_SSH_ADDRESS "docker run -d -v /data/web-app/config:/var/www/html/config -v /data/web-app/uploads:/var/www/html/uploads --restart=always -p 8080:80 $CI_REGISTRY_IMAGE --name web-app"
\end{lstlisting}
\end{latin}

\newpage

برای پیاده‌سازی از سه مرحله استفاده می‌شود:
\begin{latin}
\begin{itemize}
\item
Test
\item
Build
\item
Deploy
\end{itemize}
\end{latin}


\subsection{ \lr{Continuous Integration}}
\subsubsection{پیاده‌سازی \lr{Test}}
برای پیاده‌سازی تست، از ابزار
\lr{PHPUnit}
استفاده شده است.
اجرای
\lr{PHPUnit}
در محیط داکر صورت می‌گیرد، بدینصورت که 
ایمیج 
\texttt{php:7.4}
گرفته شده و دستورات زیر برای دانلود 
\lr{phpunit}
استفاده می‌شود:
\begin{latin}
\begin{lstlisting}[language=Bash]
curl --location --output /usr/local/bin/phpunit https://phar.phpunit.de/phpunit.phar
chmod +x /usr/local/bin/phpunit
\end{lstlisting}
\end{latin}

سپس تست‌ها انجام می‌شود:
\begin{latin}
\begin{lstlisting}[language=Bash]
phpunit --configuration phpunit.xml
\end{lstlisting}
\end{latin}

امکان بررسی کیفیت کد 
(\lr{Code Quality})
نیز با ابزارهایی مانند
\lr{SonarCube}
امکان‌پذیر است.

\subsection{ \lr{Continuous Delivery}}
\subsubsection{پیاده‌سازی \lr{Build}}
در صورت موفقیت‌آمیز بودن مرحله قبل، در اینجا باید
عملیات
\lr{Build}
ایمیج انجام شود.
در ابتدا 
شل در رجیستری که مدنظر است لاگین می‌کند. درحال حاضر امکان استفاده از رجیستری گیت‌لب (اصلی) و یا پیاده‌سازی رجیستری در گیت‌لب لوکال نیز امکان‌پذیر است.
آدرس رجیستری در متغیر محیطی\LTRfootnote{\lr{Environment Variable}}
\texttt{\$CI\_REGISTRY}
ثبت شده است.
احرازهویت بصورت خودکار و با پارامترهای
\texttt{\$CI\_ـREGISTRY\_USER}
و
\texttt{\$CI\_REGISTRY\_PASSWORD}
انجام می‌شود.

پس از لاگین، ایمیج
با استفاده از 
\texttt{Dockerfile}
بیلد شده و در مرحله بعد داخل رجیستری 
\lr{Push}
می‌شود.

\subsection{ \lr{Continuous Deployment}}
\subsubsection{پیاده‌سازی \lr{Deploy}}
این مرحله با استفاده از 
\lr{SSH}
انجام می‌شود لذا ضروری است پارامترهای
زیر در کانفیگ پروژه اضافه شوند:

\begin{itemize}
\item[-]
آدرس سرور پروداکشن:
\texttt{PROD\_SSH\_ADDRESS}
\item[-]
یوزرنیم:
\texttt{PROD\_SSH\_USER}
\item[-]
کلیدخصوصی برای اتصال سرور پروداکشن:
\texttt{PROD\_SSH\_PRIVATE\_KEY}
\end{itemize}

برای این مرحله از ایمیج
\lr{Alpine}
استفاده می‌شود، البته بسته به کانفیگ 
\lr{runner}
می‌توان
بدون ایمیج هم فرآیند را انجام داد، اما بدلیل تاثیرات جانبی توصیه نمی‌شود.

زمانی که این مرحله شروع بکار می‌کند، پکیج
\texttt{openssh}
نصب شده و کلید خصوصی اتصال به پروداکشن کپی و کلید عمومی سرور
با استفاده از 
\texttt{ssh-keyscan}
 ذخیره می‌شود.
 
برای مراحل بعدی، هربار یک کانکشن
\lr{SSH}
به سرور پروداکشن زده  و دستورات زیر اجرا می‌شود:
\begin{itemize}
\item[-]
احراز هویت در رجیستری
\item[-]
دریافت ایمیج جدید
\item[-]
متوقف‌سازی و حذف کانتینر قبلی
\item[-]
شروع کانتینر با ایمیج جدید
\end{itemize}

لازم به ذکر برای سادگی می‌توان از 
\lr{docker-compose}
استفاده کرد. همچنین درصورت استفاده از کلاستر کوبرنتیر مراحل دپلوی ساده‌تر خواهدشد.










\end{document}